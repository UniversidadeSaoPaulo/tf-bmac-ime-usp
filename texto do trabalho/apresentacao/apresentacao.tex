% $Header: /cvsroot/latex-beamer/latex-beamer/solutions/generic-talks/generic-ornate-15min-45min.en.tex,v 1.5 2007/01/28 20:48:23 tantau Exp $

\documentclass{beamer}

\mode<presentation>
{
  \usetheme{Warsaw}
  % or ...

  \setbeamercovered{transparent}
  % or whatever (possibly just delete it)
}


\usepackage[brazil]{babel}
\usepackage[utf8]{inputenc}
\usepackage{times}
\usepackage[T1]{fontenc}


\title[Plataforma para Estudo Interativo de Métodos Econométricos] % (optional, use only with long paper titles)
{Uma Plataforma de Software para o Estudo Interativo de Métodos e Algoritmos Econométricos}

%\subtitle
%{Presentation Subtitle} % (optional)

\author % (optional, use only with lots of authors)
{Carlos Duarte do Nascimento}
% - Use the \inst{?} command only if the authors have different
%   affiliation.


\institute[Universidade de São Paulo] % (optional, but mostly needed)
{
  Instituto de Matemática e Estatística\\
  Universidade de São Paulo
%  \and
%  \inst{2}%
%  Faculdade de Economia, Administração e Contabiludade\\
%  Universidade de São Paulo}
% - Use the \inst command only if there are several affiliations.
% - Keep it simple, no one is interested in your street address.
}
\date % (optional)
{4 de Março de 2009}

%\subject{Talks}
% This is only inserted into the PDF information catalog. Can be left
% out. 



% If you have a file called "university-logo-filename.xxx", where xxx
% is a graphic format that can be processed by latex or pdflatex,
% resp., then you can add a logo as follows:

 \pgfdeclareimage[height=0.5cm]{university-logo}{logousp}
 \logo{\pgfuseimage{university-logo}}



% Delete this, if you do not want the table of contents to pop up at
% the beginning of each subsection:
%\AtBeginSubsection[]
%{
%  \begin{frame}<beamer>{Outline}
%    \tableofcontents[currentsection,currentsubsection]
%  \end{frame}
%}


% If you wish to uncover everything in a step-wise fashion, uncomment
% the following command: 

%\beamerdefaultoverlayspecification{<+->}


\begin{document}

\begin{frame}
  \titlepage
\end{frame}

\begin{frame}{Conteúdo}
  \tableofcontents
  % You might wish to add the option [pausesections]
\end{frame}


% Since this a solution template for a generic talk, very little can
% be said about how it should be structured. However, the talk length
% of between 15min and 45min and the theme suggest that you stick to
% the following rules:  

% - Exactly two or three sections (other than the summary).
% - At *most* three subsections per section.
% - Talk about 30s to 2min per frame. So there should be between about
%   15 and 30 frames, all told.

\section{Introdução}
\begin{frame}{Apresentações}
	Orientador	
			\begin{itemize}
				\item{Prof Cicely Moitinho Amaral}
			\end{itemize}			
	Banca Avaliadora
			\begin{itemize}
				\item{Prof. Claudio Possani}
				\item{Prof. Sergio Muniz Oliva Filho}
	\end{itemize}
\end{frame}

\begin{frame}{O que este trabalho \textit{não} é}
	\begin{itemize}
		\item Análise de um problema matemático
		\item Um estudo profundo de Métodos Numéricos
	\end{itemize}
\end{frame}

\section{O Problema}

\begin{frame}{O Ensino de Econometria}
\end{frame}

\begin{frame}{Um Problema Econométrico}
\end{frame}

\section{Modelo da Solução}

\begin{frame}{Proposta Funcional}
\end{frame}

\begin{frame}{Diagramas de Classe}
\end{frame}

\begin{frame}{Casos de Uso}
\end{frame}

\section{Arquitetura de Software}

\begin{frame}{Escolha da Linguagem}
\end{frame}

\begin{frame}{Compilação Dinâmica}
\end{frame}

\begin{frame}{Padrões de Projeto}
\end{frame}

\section{Demonstração (I)}
\begin{frame}{Demonstração (I)}
\end{frame}

\section{Construindo uma Aula}

\begin{frame}{Um Problema Econométrico: reotmando}
\end{frame}

\begin{frame}{Métodos Numéricos}
\end{frame}

\begin{frame}{Newton-Rhapson}
\end{frame}

\begin{frame}{Gauss-Newton}
\end{frame}

\section{Demonstração (II)}
\begin{frame}{Demonstração (II)}
\end{frame}

\section{Conclusão}

\begin{frame}{Conclusões e Continuidade}
\end{frame}

\begin{frame}{Make Titles Informative.}

  You can create overlays\dots
  \begin{itemize}
  \item using the \texttt{pause} command:
    \begin{itemize}
    \item
      First item.
      \pause
    \item    
      Second item.
    \end{itemize}
  \item
    using overlay specifications:
    \begin{itemize}
    \item<3->
      First item.
    \item<4->
      Second item.
    \end{itemize}
  \item
    using the general \texttt{uncover} command:
    \begin{itemize}
      \uncover<5->{\item
        First item.}
      \uncover<6->{\item
        Second item.}
    \end{itemize}
  \end{itemize}
\end{frame}



\section*{Summary}

\begin{frame}{Summary}

  % Keep the summary *very short*.
  \begin{itemize}
  \item
    The \alert{first main message} of your talk in one or two lines.
  \item
    The \alert{second main message} of your talk in one or two lines.
  \item
    Perhaps a \alert{third message}, but not more than that.
  \end{itemize}
  
  % The following outlook is optional.
  \vskip0pt plus.5fill
  \begin{itemize}
  \item
    Outlook
    \begin{itemize}
    \item
      Something you haven't solved.
    \item
      Something else you haven't solved.
    \end{itemize}
  \end{itemize}
\end{frame}


\end{document}


